\documentclass[a4paper,titlepage,11pt,floatssmall]{mwrep}
\usepackage[left=2.5cm,right=2.5cm,top=2.5cm,bottom=2.5cm]{geometry}
\usepackage[OT1]{fontenc}
\usepackage{polski}
\usepackage[utf8]{inputenc}
\usepackage{amsmath}
\usepackage{amssymb}
\usepackage{graphicx}
\usepackage{mathrsfs}
\usepackage{rotating}
\usepackage{pgfplots}
\usetikzlibrary{pgfplots.groupplots}

\usepackage{siunitx}

\usepackage{float}
\definecolor{szary}{rgb}{0.95,0.95,0.95}
\sisetup{detect-weight,exponent-product=\cdot,output-decimal-marker={,},per-mode=symbol,binary-units=true,range-phrase={-},range-units=single}

\SendSettingsToPgf
\title{\bf Sprawozdanie z laboratorium nr 4\\ Miernik czasu reakcji, przycisk rozpoczyna i kończy pomiar, przy użyciu mikrokontrolera z rodziny MSP430  \vskip 0.1cm}
\author{Jakub Sikora \and Konrad Winnicki \and Marcin Dolicher}
\date{\today}
\pgfplotsset{compat=1.15}	
\begin{document}


\makeatletter
\renewcommand{\maketitle}{\begin{titlepage}
		\begin{center}{\LARGE {\bf
					Wydział Elektroniki i Technik Informacyjnych}}\\
			\vspace{0.4cm}
			{\LARGE {\bf Politechnika Warszawska}}\\
			\vspace{0.3cm}
		\end{center}
		\vspace{5cm}
		\begin{center}
			{\bf \LARGE Technika Mikroprocesorowa \vskip 0.1cm}
		\end{center}
		\vspace{1cm}
		\begin{center}
			{\bf \LARGE \@title}
		\end{center}
		\vspace{2cm}
		\begin{center}
			{\bf \Large \@author \par}
		\end{center}
		\vspace*{\stretch{6}}
		\begin{center}
			\bf{\large{Warszawa, \@date\vskip 0.1cm}}
		\end{center}
	\end{titlepage}
	}
\makeatother
\maketitle

\tableofcontents


\chapter{Zadanie laboratoryjne}
\section{Polecenie}
\indent{} Celem czwartego zadania laboratoryjnego było zaprojektowanie, złożenie, zaprogramowanie i przetestowanie układu z mikrokontrolerem MSP430F16x tak aby działał on jako miernik czasu reakcji – jeden przyciski rozpoczyna pomiar, drugi kończy pomiar. Zasada działania podobna jak w przypadku zwykłych stoperów. Wartość mierzonego czasu powinna być na bieżąco wyświetlana na wyświetlaczu dynamicznym, który obowiązkowo obsługiwany jest za pomocą przerwań (jedno przerwanie  - jedno przełączenie pozycji). 

\section{Szczegółowe uwagi}
\indent Częstotliwość odświeżania powinna gwarantować wyświetlenie każdej cyfry co najmniej 50 razy na sekundę. Do otrzymania satysfakcjonującej dokładności pomiaru długości trwania impulsów wymagane jest użycie trybu CAPTURE i obliczanie interwału z uwzględnieniem odczytów licznika przy zboczu początkowym i końcowym oraz liczby przerwań TxIFG, jakie zarejestrowano pomiędzy tymi zboczami. Kod programu został odpowiednio podzielony na część realizującą obsługę sprzętową i część aplikacyjną. Interakcję z systemem wzbogaciliśmy o wyświetlanie zer przy starcie systemu i przecinka, oddzielającego liczbę milisekund od sekund. 

\chapter{Projekt licznika}

\section{Opis sprzętu}
\indent{} Dzięki strukturze mikrokontrolera MSP43016x podłączenie urządzeń peryferyjnych nie było skomplikowanym zadaniem, jednak wymagało uwagi, aby nie pomylić portów obsługujących wyświetlacz. 

\begin{figure}[th]
\centering
\includegraphics[width=\textwidth]{ideowy}
\caption{Schemat ideowy}
\end{figure}

Do portu pierwszego podłączyliśmy monostabilne wyłączniki (120\_{}IN8), które posłużyły nam jako źródło sygnałów wejściowych, odpowiednio dla przycisku start – P1.2,  stop – P1.3. Dla wyświetlacza zostały zarezerwowane porty P2.0 – P2.7 (obsługa drivera) i P3.0 – P3.7 (wybór wyświetlacza 7-segmentowego). W Timer\_{}A wykorzystujemy wszystkie trzy kanały, dwa w trybie capture i jeden w trybie compare. 

\section{Opis oprogramowania}

Do zmniejszenia poboru energii użyliśmy trybu LPM0, ponieważ tylko on gwarantował nam brak zmian częstotliwości pracy zegara SMCLK. Kod programu napisaliśmy w języku C, co było niewątpliwym ułatwieniem w porównaniu do poprzednich laboratoriów.  Pozwoliło to na lepsze uporządkowanie funkcjonalności i zadań do wykonania w naszym programie. Kod stał się bardziej przejrzysty i zrozumiały. 

\subsection{Zliczanie czasu}
\indent 

Naciśnięcie przycisku może nastąpić w momencie gdy jesteśmy na zboczu zegara, dlatego do dokładnego obliczania zmierzonego czasu zastosowaliśmy przerwania w trybie capture, dzięki którym jesteśmy w stanie określić pozycję w jakiej byliśmy na tym zboczu.  Zbocza liczników są opadające. Po uruchomieniu systemu i naciśnięciu przycisku – start, stoper zaczynał odliczanie. Zapamiętujemy miejsce na zboczu i zaczynamy zliczać ilość przepełnień licznika w celu obliczenia zmierzonego czasu. Przy naciśnięciu przycisku – stop, znowu zapamiętujemy miejsce na zboczu i używając wcześniej zapisanych danych obliczamy dokładny czas do wyświetlenia.
\begin{equation*}
n_{cykli} = t_{konc} + (n_{overflow} + 1)*n_{graniczna} - t_{pocz} 
\end{equation*} 

\subsection{Obsługa wyświetlacza}
\indent 

W celu taktowania odświeżania wyświetlacza dynamicznego zastosowaliśmy przerwania generowane w trybie \texttt{compare} z Timera A. Przerwanie generuje się co określony takt czasu, w momencie gdy wartość licznika TAR zrównuje się z wartością zapisaną w kanale TACCR0, to ustawiana jest flaga przerwania (bit CCIFG w rejestrze TACCTL0) Ustawiona częstotliwość odświeżania to 1kHz. Zasada działania wyświetlacza dynamicznego jest następująca: najpierw wybieramy wyświetlacz, który będzie podświetlony, wypisujemy wartość i podtrzymujemy ją do przyjścia kolejnego przerwania. Aby wyświetlacz sprawiał wrażenie stałego, każdy segment powinien być odświeżany z częstotliwością większą niż 50 Hz. \\

\indent Zasadniczym problemem podczas odświeżania wyświetlacza jest poprawne aktualizowanie wartości. Mając na uwadze zastosowanie aplikacyjne naszego rozwiązania zdecydowaliśmy się na liczenie czasu na dwa sposoby. Podczas aktywnego zliczania długości trwania impulsu, wyświetlana jest wartość zgrubna, obliczana z pomocą przerwań z trybu \texttt{compare}. Gdy użytkownik wciśnie przycisk \texttt{STOP} to wyświetlana jest wartość dokładna obliczana z wzoru przedstawionego w sekcji powyżej. \\
 
\indent Aby czas wykonania przerwania nie był za długi przy wpisywaniu dokładnej wartości do tablicy stosujemy dzielenie modulo przez 10, co zapewnia nam satysfakcjonujące efekty i pozwala na szybkie wykonanie przerwania. Dodatkowo wyświetlanie wartości aktualnej jest aktualizowane co drugie przerwania, żeby maksymalnie, oczywiście w granicach normy skrócić czas wykonania przerwania. 

Podczas fazy testów, za pomocą oscyloskopu sprawdziliśmy ile trwają przerwania podczas aktywnego zliczania i podczas oczekiwania na start. W trakcie oczekiwania na start, przerwanie obsługuje tylko zmianę segmentu i wartości na wyświetlaczu. Przerwanie zajmuje wtedy około 20 \%  czasu. W przypadku gdy zliczanie jest aktywne to czas obsługi przerwania wydłuża się do około 30 \% dostępnego czasu.
 
\newpage

\begin{figure}[H]
\centering
\includegraphics[width=\textwidth]{przerwania_bez.jpg}
\caption{Oscylogram przedstawiający czas trwania przerwań bez aktywnego zliczania}
\end{figure}

\bigskip

\begin{figure}[H]
\centering
\includegraphics[width=\textwidth]{przerwania_z.jpg}
\caption{Oscylogram przedstawiający czas trwania przerwań z aktywnego zliczania}
\end{figure}


\newpage
\subsection{Kod programu}
\noindent
Poniżej znajduje się kod programu napisanego w języku C:\\

\noindent\#include $<$msp430.h$>$\\
\#include $<$stdio.h$>$\\
\#include $<$stdint.h$>$\\

\noindent\#define bool  uint\_{}t\\
\#define true 1\\
\#define false 0\\

// Bity sterujace driverem 7-SEG LED\\
\#define LED\_{}DP (1<<7)\\
\#define LED\_{}LT (1<<6)\\
\#define LED\_{}BI (1<<5)\\
\#define LED\_{}RBI (1<<4)\\

// Ilosc wyswietlanych znakow\\
\#define SEG\_{}LED\_{}NUMBER 7\\
\#define SEG\_{}LED\_{}DOT\_{}POSITION 5\\
\#define SEG\_{}LED\_{}DISPLAY\_{}SYSTEM 10\\

// Watrość rejestru licznika\\
\#define TACCR0\_{}VALUE 720\\

\noindent volatile char buffer[SEG\_{}LED\_{}NUMBER]; // BUFOR ZLICZANIA CZASU REAKCJI\\
volatile uint32\_{}t count\_{}time = 0; // wartosc czasu reakcji\\
volatile bool if\_{}counting = false;\\
volatile bool end\_{}of\_{}counting=false;\\


\noindent typedef struct captured\_{}cycles\\
\{\\
    int start\_{}time;\\
    int num\_{}of\_{}overflows;\\
    int end\_{}time;\\
\} captured\_{}cycles;\\

\noindent int get\_{}cycles(captured\_{}cycles time)\\
\{\\
    return (time.num\_{}of\_{}overflows + 1)*TACCR0\_{}VALUE + time.end\_{}time - time.start\_{}time;\\
\}\\

\noindent int get\_{}time(int cycles)\\
\{\\
    return (int)(cycles/10);\\
\}\\

\noindent void temp\_{}time()\\
\{\\
    uint16\_{}t n = SEG\_{}LED\_{}DOT\_{}POSITION-3; // zgrubny czas z dokładnością milisekundy\\
    buffer[n]++; // odśwież pozycję milisekund\\
    while( n<SEG\_{}LED\_{}NUMBER )\\
    \{\\
        if( buffer[n]>=SEG\_{}LED\_{}DISPLAY\_{}SYSTEM )\\
        \{\\
            buffer[n]=0;\\
            buffer[n+1]++;\\
            n++;\\
       \}\\
        else\\
            break;\\
    \}\\
\}\\

\noindent void copy\_{}buffer(char disp\_{}buffer[])\\
\{\\
	uint16\_{}t n=0;\\
    while( n<SEG\_{}LED\_{}NUMBER )\\
    \{\\
        disp\_{}buffer[n] = buffer[n]; // buffer mamy globalny\\
        n++;\\
    \}\\
\}\\

\noindent void display(uint16\_{}t disp, char disp\_{}buffer[])\\
\{\\
	//static uint16\_{}t disp = 0;\\
P3OUT = ~(1<<disp); //aktywujemy kolejny wyswietlacz\\
P2OUT = ((disp\_{}buffer[disp])\&0x0F) $\vert$ LED\_{}RBI $\vert$ LED\_{}BI $\vert$ LED\_{}LT $\vert$ LED\_{}DP; // wyswietlenie cyfry oraz bity sterujace\\
if(disp == SEG\_{}LED\_{}DOT\_{}POSITION) P2OUT \&= ~LED\_{}DP; // zapalenie kropki na odpowiedniej pozycji\\
//disp++; // wybor kolejnego wyswietlacza\\
//if(disp >= SEG\_{}LED\_{}NUMBER) disp = 0;\\
\}\\

\noindent void prepare\_{}to\_{}display\_{}ideal\_{}value(captured\_{}cycles cap\_{}time)\\
\{\\
    	int locked\_{}time = get\_{}time(get\_{}cycles(cap\_{}time)); //zatrzaskujemy wartosc na czas sekwencji wyswietlania\\

        uint\_{}t n=0;\\
		while(n<SEG\_{}LED\_{}NUMBER)\\
		\{\\
			buffer[n]=locked\_{}time \% 10;\\
			locked\_{}time /= 10;\\
			n++;\\
		\}\\
\}\\

\noindent volatile captured\_{}cycles cap\_{}time = \{0, 0, 0\};\\



\noindent int main(void)\\
\{\\
	// wylaczenie watchdoga\\
    WDTCTL = WDTPW + WDTHOLD; \\

    // inicjalizacja portu P1\\
    P1SEL = (BIT2 + BIT3); // ustaw P1.0 i P1.7 jako wejscia timera\\
    P1DIR \&= ~(BIT2 + BIT3); // ustaw jako wejscia\\
    // piny portu 1 służące do debugowania zależności czasowych przy pomocy oscyloskopu\\
    P1DIR $\vert$= BIT7 $\vert$ BIT6;\\
    P1OUT $\vert$= BIT7 $\vert$ BIT6;\\
	
    // inicjalizacja portu P2\\
    P2SEL = 0x00; // ustaw caly port 2 jako GPIO\\
    P2DIR = 0xFF; // ustaw port 2 jako wyjscia\\
    P2OUT =  (0x08\&0x0F) $\vert$ LED\_{}RBI $\vert$ LED\_{}BI $\vert$ LED\_{}LT /*$\vert$ LED\_{}DP*/; // wyswietlenie ósemki z kropką\\

    // inicjalizacja portu P3\\
    P3SEL = 0x00; // ustaw caLy port 3 jako GPIO\\
    P3DIR = 0xFF; // ustaw port 3 jako wyjscia\\
    P3OUT = 0x00; // aktywuj wszystkie wyświetlacze\\

    // Inicjacja kanalu 0, compare\\
    TACCR0 = TACCR0\_{}VALUE; // 1 kHz\\
    TACCTL0 = CCIE;         // CCR0 interrupt enabled -OK -zalaczamy przerwania do timerow (jak nie zadziala wlaczac osobna TA i TB)\\
    // Inicjacja kanalu 1, capture, przycisk start\\
    TACCTL1 =  CM\_{}2 + SCS + CAP + CCIE + CCIS\_{}0;  // falling edge, synchronus capture, CCI1A\\
    // Inicjacja kanalu 2, capture, przycisk stop\\
    TACCTL2 =  CM\_{}0 + SCS + CAP + CCIE + CCIS\_{}0;  // falling edge, synchronus capture, CCI2A\\
    // Inicjacja Timera A\\
    TACTL = TASSEL\_{}2 + MC\_{}1 + ID\_{}0 + TAIE;         // SMCLK/8, upmode -OK SMCLK = 1MHz; MC\_{}1 - UP MODE; ID\_{}0 - dzielnik /1 (nie musi byc, ale niech bedzie)\\

    \_{}bis\_{}SR\_{}register(CPUOFF+GIE); // LPM0, globalne wlaczenie obslugi przerwan\\

    while(true)\\
    \{\\

    \}\\

    return 0;\\
\}\\

\noindent \#pragma vector=TIMERA0\_{}VECTOR   // TIMERA0\_{}VECTOR - wektor do obslugi compare, kanal 0\\
\_{}\_{}interrupt void timerA0\_{}ISR(void) // timer 1kHz\\
\{\\
    cap\_{}time.num\_{}of\_{}overflows++;\\
    static uint16\_{}t disp = 0; //wybór aktywnego wyświetlacza\\
    static uint16\_{}t led\_{}timer = 0; //timer programowy wyświetlacza dynamicznego\\
    static char disp\_{}buffer[SEG\_{}LED\_{}NUMBER]; // Bufor wyswietlanych znakow\\

	if(end\_{}of\_{}counting==true )\\
	\{\\
		prepare\_{}to\_{}display\_{}ideal\_{}value(cap\_{}time);\\
		//display(disp, disp\_{}buffer);\\
		end\_{}of\_{}counting = false;\\
	\}\\

	if( if\_{}counting==true )\\
	\{\\
		temp\_{}time();\\

		if(disp==0)\\
		\{\\
			copy\_{}buffer(disp\_{}buffer);\\
		\}\\
	\}\\

	if(led\_{}timer==0) // start sekwencji wyswietlania, 1kHz/2\\
	\{\\
		led\_{}timer=1;\\
		display(disp, disp\_{}buffer);\\
		disp++;\\
		if(disp>= SEG\_{}LED\_{}NUMBER) disp = 0\\;
	\}else\\
		led\_{}timer--;\\

    //P1OUT \&= ~0x80;\\
\}\\

\noindent \#pragma vector=TIMERA1\_{}VECTOR   // TIMERA1\_{}VECTOR - wektor do obslugi capture, kanal 1, 2\\
\_{}\_{}interrupt void timerA1\_{}ISR(void)\\
\{\\
    switch(TAIV) //odczyt TAIV\\
    \{\\
        case 2: //flaga CCIFG\\
            P1OUT $\vert$= 0x40;\\
        	if\_{}counting = true;\\
            cap\_{}time.start\_{}time = TACCR1;\\
            cap\_{}time.num\_{}of\_{}overflows = 0;\\
            cap\_{}time.end\_{}time = 0;\\

            TACCTL1 \&= ~CM\_{}2;\\
            TACCTL2 \&= ~CM\_{}0;\\
            TACCTL1 $\vert$= CM\_{}0;\\
            TACCTL2 $\vert$= CM\_{}2;\\

			//end\_{}of\_{}counting=false;\\

			// zerownie buffer\\
			uint16\_{}t n=0;\\
			while(n<SEG\_{}LED\_{}DOT\_{}POSITION)\\
			\{\\
				buffer[n] = 0;\\
				n++\\
			\}\\

            break; //zrodlo TACCR1\\

        case 4: //flaga CCIFG\\
            P1OUT $\vert$= 0x40;\\
        	if\_{}counting = false;\\
            cap\_{}time.end\_{}time = TACCR2;\\

            TACCTL1 \&= ~CM\_{}0;\\
            TACCTL2 \&= ~CM\_{}2;\\
            TACCTL1 $\vert$ = CM\_{}2;\\
            TACCTL2 $\vert$= CM\_{}0;\\


			end\_{}of\_{}counting=true;\\
            break; //zrodlo TACCR2\\
    \}\\
\}\\
       
\end{document}